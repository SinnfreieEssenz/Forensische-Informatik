\chapter{Prolog}

\section{Beweismittel}

\subsection{Identifikation}

TODO access LOg Konfigurationsdatei
Die Beweismittel zwei Datenträger mit den Namen und SHA256-Summen\\
ex A.dd a01b1963c1c6b1636242b72ab3423e310186d554c079e288abd3ea5a9f22b55f (Asservat A) und ex B.dd ea2fb3488e11f6426a27bcee054c2629c55334f565d0ed1a2346c0d32f2c3d80 (Asservat B) wurden persönlich bei der Staatsanwaltschaft abgeholt mit Identitätskontrolle der anwesenden Personen. Ebenfalls abgeholt wurden die von der Spurensicherung erstellten Hash-Summen zur Überprüfung der Echtheit der Datenträger. Die Echtheit der Datenträger konnte somit erfolgreich verifiziert werden. Daraufhin wurde eine 1 zu 1 Kopie der Datenträger angefertigt, welche für die folgende Analyse verwendet wurden. Die orginalen Datenträger wurden bis auf die Kopie für nichts anderes benutzt. Ebenfalls wurde nur auf einem Passwort gesicherten Computer ohne Internet Anschluss in einem Raum gearbeitet, welcher nach Verwendung abgesperrt wurde. Manipulation der Beweise durch dritte ist somit ausschließbar.

\subsection{Verlauf}
Der Ermittlungsverlauf des Falles sah wie folgt aus:

	Festplatte 1:
	\begin{itemize}
		\item Nachweis der Integrität des Asservats
		\item Anfertigung einer Kopie
		\item Analyse des Datenträgers und Suche nach möglichen Beweisen
	\end{itemize}

	Festplatte 2:
	\begin{itemize}
		\item Nachweis der Integrität des Asservats
		\item Anfertigung einer Kopie
		\item Analyse des Datenträgers und Suche nach möglichen Beweisen
	\end{itemize}

\section{Auftrag}
Die Staatsanwaltschaft ermittelt zur Zeit gegen die zwei Brüder John und Frank Doe aus Erlangen. Sie stehen im Verdacht, am 8.10.2015 zwischen 23 und 24 Uhr einen Webserver an der Universität Erlangen angegriffen, kompromittiert und in der Folge Daten aus gespäht zu haben. Neben §202a StGB sieht die Staatsanwaltschaft den Tatbestand der Datenveränderung gemäß §303a StGB als erfüllt an.Beider Analyse des Webservers wurde die IP-Adresse 131.188.31.68 ermittelt. Es ist davon auszugehen, dass die Angriffe auf den Webserver von dieser IP-Adresse aus erfolgten. Übereine Abfragegemäß §100j Abs.2 StPO wurde Frank Doe als Anschlussinhaber dieser IP-Adresse zum oben genannten Zeitpunkt ermittelt. Im Rahmen einer Hausdurchsuchung wurden zwei Computersysteme vorgefunden und beschlagnahmt. Es wurde festgestellt, dass sowohl John als auch Frank Doe aber keine weiteren Personen die Wohnung bewohnen. Eine Zuordnung der beiden Rechner zu den beiden Personen war im Rahmen der Beschlagnahme nicht möglich. Beide Brüder haben ausgesagt, im fraglichen Zeitraum auf die Webseite zugegriffen zu haben. Weitere Angaben zur Sache haben sie allerdings gemäß §55 StPO nicht gemacht. Die Kriminalinspektion 5(Cybercrime und digitale Spuren) hat im Zuge der Ermittlungen bereits eine Analyse des Webservers durchgeführt. Die vorläufigen Untersuchungsergebnisse sind unten beigefügt. Die Staatsanwaltschaft erbittet Antworten auf folgende Fragen:\\
1. Können Sie die beiden Rechner den beiden Brüdern persönlich zuordnen?\\
2. Wurde von einem der beiden Rechner der Angriff durchgeführt? Fallsja, von welchem Rechner?

\section{Arbeitsumgebung}
Die komplette Untersuchung wurde ausschließlich unter folgenden Arbeitsbedingungen ausgeführt, wobei eine virtuelle Maschine benutzt wurde um externe Einflüsse zu minimieren:
\begin{itemize}
	\item Oracle VM-Virtualbox 6.0.24  mit Standardeinstellungen
	\item Kali-Linux-2021.1-vbox-amd64
	\item Host OS Windows 10 Pro Intel Core i5-4670K CPU@ 3.40GHz, x64-Bit-Betriebssystem, 8 GB RAM
	\item The Sleuth Kit ver 4.10.1
	\item TestDisk 7.1, Data Recovery Utility, July 2019
	\item PhotoRec 7.1, Data Recovery Utility, July 2019
\end{itemize}
