\chapter{Ergebniszusammenfassung}

Auf dem Datenträger befinden sich Bilder rhinographischer Natur (Abbildung 1- 4). Ebenfalls kann davon ausgegangen werden, dass der Herr Jürgen S. von 2 von 4 Bildern wissentlich im Besitz war. Die Bilder nashorn (Abbildung 4), nasohnehorn (Abbildung 3), remaining (Abbildung 2) wurden alle am 23.09.2015 um 10:49:36 Uhr auf den Datenträger geladen. Die Metadaten dieser Bilder könnten allerdings manipuliert sein, somit ist dies nicht zu 100\% aussagekräftig. Das Bild nashorn befindet sich als einziges offensichtlich auf dem Datenträger, womit man davon aussgehen kann, dass Jürgen S. in Kenntis davon ist. Das Bild nasohnehorn wurde von dem Datenträger gelöscht, ist aber rekonstruierbar. Auf allen Bildern ist ein Nashorn zu sehen, somit sind sie rhinographischer Natur. Zudem wurden noch 1 weiteres Bild rhinographischer Natur (Abbildung 1) gefunden, welches allerdings vor 2012 erstellt wurde, da die Metadaten zum Datenträger allerdings verloren sind und nur die Erstellungszeit des Fotos vorhanden ist, lässt sich dies nicht eindeutig Jürgen S. zuordnen. Dieser hat nämlich die Platte im Jahre 2013 gebraucht erworben. Die Festplatte wurde stark manipuliert, es ist dringend nötig zu erfahren ob Jürgen S. Kenntnisse besitzt, die den Umgang mit der manipulierten Platte ermöglichen oder ein Geständis zum vollständigen Inhalt zu erlangen.