\chapter{Prolog}

\section{Beweismittel}

Datenträger mit der SHA256-Summe:\\ 
1f23fcf72f931e14a2762b3014b97f51e5031c045129d044287457a996b0c4cc

\subsection{Identifikation}

??
echo "1f23fcf72f931e14a2762b3014b97f51e5031c045129d044287457a996b0c4cc exercise\_1.img" | sha256sum -c
exercise\_1.img: OK

\subsection{Verlauf}
	\begin{itemize}
	\item Anfertigen von Hashsummen
	\item Bestimmen des Partitionstyps
	\item Analyse der Partitionstabelle
	\item Analyse des Dateisystems
	\item Analyse interessanter Dateien
	\item File-Carving zum Auffinden nicht referenzierter Dateien
	\item Prüfung der Integrität des Abbilds
	\end{itemize}

\section{Auftrag}
Untersuchungsauftrag: Verdacht auf Besitz illegaler Nashornographie gemäß § 184k StGBVorbemerkung: § 184k ist ein hypothetischer Straftatbestand, der die Umständevon § 184b übungsfreundlich nachbildet. Der neue Straftatbestand verbietet denBesitz und die Weitergabe von Nashornbildern. In der Rechtspraxis wird man be-straft, wenn manmindestens dreiNashornbilder wissentlich besitzt. Die Aufgaben-stellung geht auf den DFRWS Forensic Rodeo Challenge 2005 zurück, vgl. \url{http://www.cfreds.nist.gov/dfrws/Rhino_Hunt.html}\\
Die Staatsanwaltschaft hat ein Ermittlungsverfahren gegen Herrn Jürgen S. ein-geleitet. Es besteht der Verdacht auf Besitz illegaler Nashornbilder (Nashornographie) gemäß § 184k StGB.\\
Im Rahmen einer Hausdurchsuchung am 25.10.2016 wurde in der Wohnung von Herrn S. ein Datenträger (externe USB-Festplatte Marke Seetor, Asservatennummer 35/17/2015, Baujahr 2007) beschlagnahmt. Der Beschuldigte hat zugegeben,der Besitzer des Datenträgers zu sein, welchen er 3 Jahre vor der Beschlagnahmung gebraucht im Internet erworben hatte.\\
Durch die aktuelle Überlastung der Kriminalinspektion 5 (Cybercrime und digitaleSpuren) ist eine zeitnahe Auswertung in der polizeilichen Forensik nicht möglich. Deshalb bestellt die Staatsanwaltschaft Sie als externen Gutachter/externe Gut-achterin zur Analyse des beschlagnahmten Datenträgers.\\
Die Staatsanwaltschaft erbittet Antworten auf folgende Fragen:\\
1. Befinden sich auf dem Datenträger Bilddateien, die potentiell rhinographi-scher Natur sind?\\
2. Bei wievielen der Bilder besteht Grund zur Annahme, der Beschuldigte wissevon ihrer Existenz?\\
Die Staatsanwaltschaft händigt Ihnen das Abbild des Datenträgers aus. Die SHA256-Summe lautet:\\ 1f23fcf72f931e14a2762b3014b97f51e5031c045129d044287457a996b0c4cc\\
Die Staatsanwaltschaft erwartet Ihre Ergebnisse in Form eines Untersuchungsbe-richts bis zum13.05.2021 (23:59 Uhr).

\section{Arbeitsumgebung}

Oracle VM-Virtualbox mit Kali-Linux-2021.1-vbox-amd64